\section{Conclusions}

In this experimentation we developed a system with microservices, with the aim to verify the feasibility of scaling a workload using stochastic properties of the arrival and service process. We developed a scaling system that models using Sirio framework, and uses its quantitative evaluation for an informed scaling mechanism (in our case funded on rejection rate).

We found that supposing to approximate the arrival process with an exponential form is effective even in the case of a deterministic inter-arrival time, something positive considering how easily we can analyze Markovian systems. Even considering the impossibility to use Bernstein phase types due to their high resource used, we believe that the major precision is not compensating a heavier execution.

In fact, we argue that for arrival processes with a high number of messages per second (condition  that this type of scaling are design for) the shape is not so relevant. So, it's better to have an easier model to obtain quick result to be more reactive.

Another thing to consider is that, for rejection rates target so low, practically the system prescribe a number of service replicas such that the composed rate of service matches the arrival rate. Even if this can conclude in favor of an even simpler solution, the real main advantage of a stochastic approach is the flexibility of the system. We strongly advocate for a use of the Sirio Framework oriented to transient analysis instead of steady state. With this objective in mind, it's trivial to think, as a future development, a scaler that exploits the number of elements in the queue to forecast rejection rate within a short period of time. So to lower the number of replicas even further with a low queue.

In conclusion, we show the feasibility of a horizontal scaling using informed stochastic model, placing the base for further developments.