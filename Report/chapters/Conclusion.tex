\section{Conclusions}

In this experiment, we developed a microservices-based system with the aim of verifying the feasibility of workload scaling using the stochastic properties of the arrival and service processes. We developed a scaling system that models a STPN using the Sirio framework and uses its quantitative evaluation for an informed scaling mechanism (in our case based on the rejection rate).

We have found that, assuming we approximate the arrival process with an exponential form, this is also effective in the case of a deterministic time between arrivals, which is positive considering the ease with which we can analyse Markovian systems. Even considering the impossibility of using Bernstein phase types due to high resource consumption, we believe that the increased accuracy does not compensate for heavier execution.

In fact, we argue that for arrival processes with a high number of messages per second (condition that this type of scaling are design for) the shape is not so relevant. So, it's better to have an easier model to obtain quick result and be more reactive.

Another aspect to consider is that, for such low rejection rates, the system practically prescribes a number of service replications such that the composite service rate corresponds to the arrival rate. Even if this can conclude in favor of an even simpler solution, the real main advantage of a stochastic approach is the flexibility of the system. We strongly advocate for a use of the Sirio Framework oriented to transient analysis instead of steady state. With this goal in mind, it is trivial to think, as a future development, of a scaler that uses the number of elements in the queue to predict the rejection rate over a short period of time. Thus, to further reduce the number of replicas with a low queue, while increasing processing on a nearly full queue.

In conclusion, we have demonstrated the feasibility of horizontal resizing using an informed stochastic model, laying the foundations for further developments, for example based on machine learning.
